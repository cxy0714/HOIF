% Options for packages loaded elsewhere
\PassOptionsToPackage{unicode}{hyperref}
\PassOptionsToPackage{hyphens}{url}
\documentclass[
]{article}
\usepackage{xcolor}
\usepackage[margin=1in]{geometry}
\usepackage{amsmath,amssymb}
\setcounter{secnumdepth}{-\maxdimen} % remove section numbering
\usepackage{iftex}
\ifPDFTeX
  \usepackage[T1]{fontenc}
  \usepackage[utf8]{inputenc}
  \usepackage{textcomp} % provide euro and other symbols
\else % if luatex or xetex
  \usepackage{unicode-math} % this also loads fontspec
  \defaultfontfeatures{Scale=MatchLowercase}
  \defaultfontfeatures[\rmfamily]{Ligatures=TeX,Scale=1}
\fi
\usepackage{lmodern}
\ifPDFTeX\else
  % xetex/luatex font selection
\fi
% Use upquote if available, for straight quotes in verbatim environments
\IfFileExists{upquote.sty}{\usepackage{upquote}}{}
\IfFileExists{microtype.sty}{% use microtype if available
  \usepackage[]{microtype}
  \UseMicrotypeSet[protrusion]{basicmath} % disable protrusion for tt fonts
}{}
\makeatletter
\@ifundefined{KOMAClassName}{% if non-KOMA class
  \IfFileExists{parskip.sty}{%
    \usepackage{parskip}
  }{% else
    \setlength{\parindent}{0pt}
    \setlength{\parskip}{6pt plus 2pt minus 1pt}}
}{% if KOMA class
  \KOMAoptions{parskip=half}}
\makeatother
\usepackage{graphicx}
\makeatletter
\newsavebox\pandoc@box
\newcommand*\pandocbounded[1]{% scales image to fit in text height/width
  \sbox\pandoc@box{#1}%
  \Gscale@div\@tempa{\textheight}{\dimexpr\ht\pandoc@box+\dp\pandoc@box\relax}%
  \Gscale@div\@tempb{\linewidth}{\wd\pandoc@box}%
  \ifdim\@tempb\p@<\@tempa\p@\let\@tempa\@tempb\fi% select the smaller of both
  \ifdim\@tempa\p@<\p@\scalebox{\@tempa}{\usebox\pandoc@box}%
  \else\usebox{\pandoc@box}%
  \fi%
}
% Set default figure placement to htbp
\def\fps@figure{htbp}
\makeatother
\setlength{\emergencystretch}{3em} % prevent overfull lines
\providecommand{\tightlist}{%
  \setlength{\itemsep}{0pt}\setlength{\parskip}{0pt}}
\usepackage{bookmark}
\IfFileExists{xurl.sty}{\usepackage{xurl}}{} % add URL line breaks if available
\urlstyle{same}
\hypersetup{
  pdftitle={HOIF.R},
  pdfauthor={thinkbook-cxy},
  hidelinks,
  pdfcreator={LaTeX via pandoc}}

\title{HOIF.R}
\author{thinkbook-cxy}
\date{2026-01-26}

\begin{document}
\maketitle

\#' HOIF Estimators for Average Treatment Effect \#' \#' This file
implements the Higher Order Influence Function (HOIF) estimators \#' for
Average Treatment Effect (ATE) estimation with nuisance functions. \#'
\#' @author Xingyu Chen \#' @date 2026-01-23

\section{Required packages}\label{required-packages}

\section{install.packages(c(``splines'',
``corpcor''))}\label{install.packagescsplines-corpcor}

\#' Convert nested list expression to Einstein notation \#' \#' Converts
a nested list expression like {[}{[}1,2{]},{[}2,3{]},{[}3,4{]}{]} to
Einstein \#' notation like ``ab,bc,cd-\textgreater{}''. \#' \#' @param
expr\_list A list of integer vectors, each of length 2 \#' \#' @return
Character string in Einstein notation \#' @keywords internal \#' \#'
@examples \#' \dontrun{
#' expr_list_to_einstein(list(c(1,2), c(2,3), c(3,4)))  # Returns "ab,bc,cd->"
#' } expr\_list\_to\_einstein \textless- function(expr\_list) \{ \#
Collect and SORT indices all\_indices \textless-
sort(unique(as.integer(unlist(expr\_list)))) n\_unique \textless-
length(all\_indices)

if (n\_unique \textgreater{} 26) \{ stop(``Too many unique indices
(\textgreater26)'') \}

index\_to\_letter \textless- setNames(letters{[}seq\_len(n\_unique){]},
all\_indices)

\# Convert each tensor terms \textless- vapply(expr\_list, function(idx)
\{ idx \textless- as.integer(idx)

\begin{verbatim}
if (length(idx) == 1) {
  index_to_letter[as.character(idx)]

} else if (length(idx) == 2) {
  paste0(index_to_letter[as.character(idx[1])],
         index_to_letter[as.character(idx[2])])

} else {
  stop("Only rank-1 or rank-2 tensors supported")
}
\end{verbatim}

\}, character(1))

paste0(paste(terms, collapse = ``,''), ``-\textgreater{}'') \}

\#' Compute a Higher-Order U-Statistic \#' \#' Computes a higher-order
U-statistic given pre-computed kernel matrices \#' using either an
Einstein summation expression or nested list notation. \#' This function
calls Python's u-stats package via reticulate. \#' \#' \textbf{Note}:
This function requires Python and the u-stats package to be installed.
\#' Run \code{setup_hoif()} to install the required dependencies. \#'
\#' @param tensors A list of numeric matrices/vectors of equal
dimensions. \#' @param expression Either a character string (Einstein
notation like ``ab,bc-\textgreater{}'') \#' or a nested list (like
list(c(1,2), c(2,3))). Can also accept the special \#' format list(1,
list(1,2), \ldots, j) from compute\_hoif\_estimators. \#' @param backend
Character string, either ``numpy'' or ``torch'' (default). \#' @param
average Logical; whether to return the averaged U-statistic (default
TRUE). \#' \#' @return A numeric scalar. \#' \#' @examples \#' \dontrun{
#' # First, setup the Python environment
#' setup_hoif()
#'
#' # Then use the function
#' v1 c- runif(100)
#' H1 <- matrix(runif(100), 10, 10)
#' H2 <- matrix(runif(100), 10, 10)
#' ustat(list(H1, H2), "ab,bc->")
#' # Or equivalently:
#' ustat(list(v1, H1, H2),"a,ab,bc->")
#' } \#' \#' @export ustat \textless- function(tensors, expression,
backend = c(``torch'', ``numpy''), average = TRUE) \{

if (!check\_python\_env()) \{ stop( ``Python environment not properly
configured.\n'', ``Please run: setup\_hoif()\n'', ``Or check setup
status with: check\_hoif\_setup()'', call. = FALSE ) \}

backend \textless- match.arg(backend)

if (backend == ``torch'' \&\&
!reticulate::py\_module\_available(``torch'')) \{ warning( ``Torch
backend not available; falling back to numpy.\n'', ``Install torch with:
reticulate::py\_install(`torch')'', call. = FALSE ) backend \textless-
``numpy'' \}

ustats \textless- tryCatch(\{ reticulate::import(``u\_stats'',
delay\_load = TRUE) \}, error = function(e) \{ stop( ``Failed to import
u\_stats module.\n'', ``Please run: setup\_hoif()\n'', ``Error:'',
e\$message, call. = FALSE ) \})

ustats\$set\_backend(backend) np \textless-
reticulate::import(``numpy'', convert = FALSE)

\# ========================================================== \# 🔁
Expression auto-conversion \#
========================================================== if
(is.list(expression)) \{

\begin{verbatim}
# 支持 build_Ej() 生成的格式: list(1, c(1,2), ..., j)
valid_structure <- all(vapply(expression, function(x) {
  is.numeric(x) && length(x) %in% c(1, 2)
}, logical(1)))

if (!valid_structure) {
  stop("Expression list must contain numeric vectors of length 1 or 2.",
       call. = FALSE)
}

expression <- expr_list_to_einstein(expression)
\end{verbatim}

\} else if (!is.character(expression)) \{ stop(``Expression must be
either a character string or a nested list'', call. = FALSE) \}

\# ========================================================== \# 🔁
Tensor auto-conversion \#
========================================================== tensors
\textless- lapply(tensors, function(x) \{

\begin{verbatim}
# ✅ 已经是 Python numpy / torch 张量 → 直接放行
if (inherits(x, "python.builtin.object")) {
  return(x)
}

# ❌ 必须是 numeric
if (!is.numeric(x)) {
  stop("All tensors must be numeric (vector, matrix, or array).",
       call. = FALSE)
}

# 🔹 R 向量(无 dim) → 1D numpy array
if (is.null(dim(x))) {
  return(np$array(as.numeric(x), dtype = "float32"))
}

# 🔹 任意维 R array → 保持维度转 numpy
return(np$array(x, dtype = "float32"))
\end{verbatim}

\})

\# ========================================================== \# 🚀 Call
Python ustat \#
========================================================== result
\textless- tryCatch(\{ ustats\(ustat(
      tensors = tensors,
      expression = expression,
      average = average
    )
  }, error = function(e) {
    stop(
      "Error computing U-statistic: ", e\)message, ``\n'', ``Expression
used:'', expression, call. = FALSE ) \})

as.numeric(result) \}

\#' Transform covariates to basis functions \#' \#' @param X Matrix of
covariates (n x p) \#' @param method Character: ``splines'' or
``fourier'' \#' @param k Integer: dimension of basis expansion \#'
@param degree Integer: degree for B-splines (default 3) \#' @param
period Numeric: period for Fourier basis (default 1) \#' \#' @return
Matrix Z (n x k) of transformed covariates with intercept \#' @export
transform\_covariates \textless- function(X, method = ``splines'', k =
10, degree = 3, period = 1) \{ if (is.vector(X)) X \textless- matrix(X,
ncol = 1) n \textless- nrow(X) p \textless- ncol(X)

\# Scale X to {[}0, 1{]} for each dimension X\_scaled \textless-
apply(X, 2, function(x) \{ (x - min(x)) / (max(x) - min(x) + 1e-10) \})

\# Initialize with intercept Z \textless- matrix(1, nrow = n, ncol = 1)

\# Generate basis for each dimension separately if (method ==
``splines'') \{ for (j in 1:p) \{ \# B-splines basis knots \textless-
seq(0, 1, length.out = k - degree + 1) basis\_j \textless-
splines::bs(X\_scaled{[}, j{]}, knots = knots{[}-c(1, length(knots)){]},
degree = degree, intercept = FALSE) Z \textless- cbind(Z, basis\_j) \}
\} else if (method == ``fourier'') \{ for (j in 1:p) \{ \# Fourier
basis: cos and sin terms freq \textless- 1:floor(k / 2) for (f in freq)
\{ Z \textless- cbind(Z, cos(2 * pi * f * X\_scaled{[}, j{]} / period))
Z \textless- cbind(Z, sin(2 * pi * f * X\_scaled{[}, j{]} / period)) \}
\} \} else \{ stop(``Method must be `splines' or `fourier'\,'') \}

return(Z) \}

\#' Compute residuals for both treatment groups \#' \#' @param A
Treatment vector (0 or 1) \#' @param Y Outcome vector \#' @param mu1
Predicted outcomes under treatment (mu(1, X)) \#' @param mu0 Predicted
outcomes under control (mu(0, X)) \#' @param pi Propensity scores \#'
\#' @return List with R1, r1, R0, r0 \#' @export compute\_residuals
\textless- function(A, Y, mu1, mu0, pi) \{ \# For a = 1 R1 \textless- A
* (Y - mu1) r1 \textless- 1 - A / pi

\# For a = 0 R0 \textless- (1 - A) * (Y - mu0) r0 \textless- 1 - (1 - A)
/ (1 - pi)

return(list(R1 = R1, r1 = r1, R0 = R0, r0 = r0)) \}

\#' Compute inverse of weighted Gram matrix \#' \#' @param Z Basis
matrix (n x k) \#' @param A Treatment vector \#' @param method
Character: ``direct'', ``nlshrink'', or ``corpcor'' \#' \#' @return List
with Omega1 and Omega0 (inverse Gram matrices) \#' @export
compute\_gram\_inverse \textless- function(Z, A, method = ``direct'') \{
n \textless- nrow(Z) k \textless- ncol(Z)

\# Compute weights s1 \textless- A \# s\_i\^{}1 = A\_i s0 \textless- 1 -
A \# s\_i\^{}0 = 1 - A\_i

\# Weighted Gram matrices using eigenMapMatMult if available if
(requireNamespace(``sumt'', quietly = TRUE)) \{ \# Use faster matrix
multiplication Z\_weighted1 \textless- Z * sqrt(s1) Z\_weighted0
\textless- Z * sqrt(s0) G1 \textless-
sumt::eigenMapMatMult(t(Z\_weighted1), Z\_weighted1) / n G0 \textless-
sumt::eigenMapMatMult(t(Z\_weighted0), Z\_weighted0) / n \} else \{ \#
Fallback to standard crossprod G1 \textless- crossprod(Z * sqrt(s1)) / n
G0 \textless- crossprod(Z * sqrt(s0)) / n \}

\# Compute inverses if (method == ``direct'') \{ Omega1 \textless-
tryCatch(\{ chol2inv(chol(G1)) \}, error = function(e) \{ warning(``G1
not positive definite, using Moore-Penrose inverse'') MASS::ginv(G1) \})

\begin{verbatim}
Omega0 <- tryCatch({
  chol2inv(chol(G0))
}, error = function(e) {
  warning("G0 not positive definite, using Moore-Penrose inverse")
  MASS::ginv(G0)
})
\end{verbatim}

\} else if (method == ``nlshrink'') \{ if
(!requireNamespace(``corpcor'', quietly = TRUE)) \{ stop(``Package
`corpcor' needed for nlshrink method'') \} Omega1 \textless-
corpcor::invcov.shrink(G1, verbose = FALSE) Omega0 \textless-
corpcor::invcov.shrink(G0, verbose = FALSE) \} else if (method ==
``corpcor'') \{ if (!requireNamespace(``corpcor'', quietly = TRUE)) \{
stop(``Package `corpcor' needed for corpcor method'') \} Omega1
\textless- corpcor::pseudoinverse(G1) Omega0 \textless-
corpcor::pseudoinverse(G0) \} else \{ stop(``Method must be `direct',
`nlshrink', or `corpcor'\,'') \}

return(list(Omega1 = Omega1, Omega0 = Omega0)) \}

\#' Compute basis projection matrices \#' \#' @param Z Basis matrix (n x
k) \#' @param Omega1 Inverse Gram matrix for treatment group \#' @param
Omega0 Inverse Gram matrix for control group \#' \#' @return List with
B1 and B0 (projection matrices) \#' @export compute\_basis\_matrix
\textless- function(Z, Omega1, Omega0) \{ B1 \textless- Z \%\emph{\%
Omega1 \%}\% t(Z) B0 \textless- Z \%\emph{\% Omega0 \%}\% t(Z)

return(list(B1 = B1, B0 = B0)) \}

\#' Convert nested list expression to Einstein notation \#' \#' Converts
a nested list expression like {[}{[}1,2{]},{[}2,3{]},{[}3,4{]}{]} to
Einstein \#' notation like ``ab,bc,cd-\textgreater{}''. \#' \#' @param
expr\_list A list of integer vectors, each of length 2 \#' \#' @return
Character string in Einstein notation \#' @keywords internal \#' \#'
@examples \#' \dontrun{
#' expr_list_to_einstein([[1,2],[2,3],[3,4]])  # Returns "ab,bc,cd->"
#' } expr\_list\_to\_einstein \textless- function(expr\_list) \{ \# Step
1: Collect all unique indices (as characters for safe naming)
all\_indices \textless- sort(unique(unlist(expr\_list)))

if (length(all\_indices) \textgreater{} 26) \{ stop(``Too many unique
indices (\textgreater26); Einstein notation limited to a-z'') \}

\# Step 2: Map each index to a letter: 1-\textgreater a,
2-\textgreater b, \ldots, 26-\textgreater z \# Use as.character() to
avoid numeric name issues index\_to\_letter \textless-
setNames(letters{[}seq\_along(all\_indices){]},
as.character(all\_indices))

\# Step 3: Convert each group (e.g., c(1,2,3) -\textgreater{} ``abc'')
terms \textless- sapply(expr\_list, function(indices) \{ \# Ensure
indices is a vector (even if length 1) idx\_chars \textless-
as.character(indices) letters\_vec \textless-
index\_to\_letter{[}idx\_chars{]}

\begin{verbatim}
# Safety check: any missing?
if (any(is.na(letters_vec))) {
  stop("Found unmapped index in: ", paste(indices, collapse = ", "))
}

paste0(letters_vec, collapse = "")
\end{verbatim}

\})

\# Step 4: Join with commas and append ``-\textgreater{}''
paste0(paste(terms, collapse = ``,''), ``-\textgreater{}'') \}

\#' @param j Integer \textgreater= 2 \#' @return Nested list
representing {[}1,{[}1,2{]},\ldots,{[}j-1,j{]},j{]} \#' @export
build\_Ej \textless- function(j) \{ if (j \textless{} 2) stop(``j must
be \textgreater= 2'')

E\_j \textless- vector(``list'', j + 1)

\# First single index tensor: {[}1{]} E\_j{[}{[}1{]}{]} \textless- 1

\# Middle B tensors: {[}1,2{]}, {[}2,3{]}, \ldots, {[}j-1,j{]} for (k in
1:(j-1)) \{ E\_j{[}{[}k + 1{]}{]} \textless- c(k, k + 1) \}

\# Last single index tensor: {[}j{]} E\_j{[}{[}j + 1{]}{]} \textless- j

E\_j \}

\#' Compute HOIF estimators for ATE \#' \#' @param residuals List with
R1, r1, R0, r0 \#' @param B\_matrices List with B1 and B0 \#' @param m
Maximum order \#' @param backend Character: ``torch'' (default) or
``numpy'' \#' \#' @return List with ATE, HOIF, and IIFF estimates for
each order \#' @export compute\_hoif\_estimators \textless-
function(residuals, B\_matrices, m = 5, backend = ``torch'') \{ \#
Initialize storage U1 \textless- numeric(m) U0 \textless- numeric(m)
IIFF1 \textless- numeric(m) IIFF0 \textless- numeric(m) HOIF1 \textless-
numeric(m) HOIF0 \textless- numeric(m) ATE \textless- numeric(m)

\# Extract components R1 \textless- residuals\(R1
  r1 <- residuals\)r1 R0 \textless- residuals\(R0
  r0 <- residuals\)r0 B1 \textless- B\_matrices\(B1
  B0 <- B_matrices\)B0

\# Compute U-statistics for each order j = 2 to m for (j in 2:m) \{ \#
Construct tensor list T\_j\^{}a \# T\_j\^{}a = list(R\^{}a, B\^{}a,
B\^{}a, \ldots, B\^{}a (j-1 times), r\^{}a) T\_j\_1 \textless- list(R1)
for (k in 1:(j-1)) \{ T\_j\_1{[}{[}k+1{]}{]} \textless- B1 \}
T\_j\_1{[}{[}j+1{]}{]} \textless- r1

\begin{verbatim}
T_j_0 <- list(R0)
for (k in 1:(j-1)) {
  T_j_0[[k+1]] <- B0
}
T_j_0[[j+1]] <- r0

# Construct expression E_j^a directly as nested list
# For j tensors: [1,[1,2], [2,3], ..., [j-1,j],j]
# This represents: R_i * B_{i,i1} * B_{i1,i2} * ... * r_{ij}
E_j <- build_Ej(j)

# Compute U-statistics using ustat function
U1[j] <- (-1)^j * ustat(tensors = T_j_1, expression = E_j,
                        backend = backend, average = TRUE)
U0[j] <- (-1)^j * ustat(tensors = T_j_0, expression = E_j,
                        backend = backend, average = TRUE)
\end{verbatim}

\}

\# Compute IIFF and HOIF for each order l = 2 to m for (l in 2:m) \{ \#
IIFF\_l = sum\_\{j=2\}\^{}l C\_j\^{}l * U\_j \# where C\_j\^{}l =
choose(l-2, l-j) for (j in 2:l) \{ C\_jl \textless- choose(l - 2, l - j)
IIFF1{[}l{]} \textless- IIFF1{[}l{]} + C\_jl * U1{[}j{]} IIFF0{[}l{]}
\textless- IIFF0{[}l{]} + C\_jl * U0{[}j{]} \}

\begin{verbatim}
# HOIF_l = sum_{j=2}^l IIFF_j
HOIF1[l] <- sum(IIFF1[2:l])
HOIF0[l] <- sum(IIFF0[2:l])

# ATE_l = HOIF_l^1 - HOIF_l^0
ATE[l] <- HOIF1[l] - HOIF0[l]
\end{verbatim}

\}

\# Return results for orders 2 to m return(list( ATE = ATE{[}2:m{]},
HOIF1 = HOIF1{[}2:m{]}, HOIF0 = HOIF0{[}2:m{]}, IIFF1 = IIFF1{[}2:m{]},
IIFF0 = IIFF0{[}2:m{]}, orders = 2:m )) \}

\#' Main function: HOIF estimators for ATE with optional sample
splitting \#' \#' @param X Covariate matrix (n x p) \#' @param A
Treatment vector (n x 1) \#' @param Y Outcome vector (n x 1) \#' @param
mu1 Predicted outcomes under treatment \#' @param mu0 Predicted outcomes
under control \#' @param pi Propensity scores \#' @param
transform\_method Character: ``splines'' or ``fourier'' \#' @param k
Dimension of basis expansion \#' @param inverse\_method Character:
``direct'', ``nlshrink'', or ``corpcor'' \#' @param m Maximum order for
HOIF \#' @param sample\_split Logical: whether to use sample splitting
\#' @param K Number of folds for sample splitting (if used) \#' @param
backend Character: ``torch'' (default) or ``numpy'' \#' @param seed
Random seed for reproducibility (for sample splitting) \#' @param
\ldots{} Additional arguments passed to transform\_covariates \#' \#'
@return List with ATE, HOIF, and IIFF estimates \#' @export hoif\_ate
\textless- function(X, A, Y, mu1, mu0, pi, transform\_method =
``splines'', k = 10, inverse\_method = ``direct'', m = 5, sample\_split
= FALSE, K = 5, backend = ``torch'', seed = NULL, \ldots) \{

n \textless- length(Y)

\# Step 1: Transform covariates (done on full data) Z \textless-
transform\_covariates(X, method = transform\_method, k = k, \ldots)

\# Step 2: Compute residuals (done on full data) residuals \textless-
compute\_residuals(A, Y, mu1, mu0, pi)

if (!sample\_split) \{ \# No sample splitting: standard procedure

\begin{verbatim}
# Step 3: Compute inverse Gram matrices
Omega <- compute_gram_inverse(Z, A, method = inverse_method)

# Step 4: Compute basis matrices
B_matrices <- compute_basis_matrix(Z, Omega$Omega1, Omega$Omega0)

# Step 5: Compute HOIF estimators
results <- compute_hoif_estimators(residuals, B_matrices, m, backend)
\end{verbatim}

\} else \{ \# Sample splitting (cross-fitting)

\begin{verbatim}
# Set seed for fold creation if provided
if (!is.null(seed)) {
  set.seed(seed)
}

# Create fold indices
fold_indices <- sample(rep(1:K, length.out = n))

# Storage for fold-specific estimates
ATE_folds <- matrix(0, nrow = K, ncol = m - 1)
HOIF1_folds <- matrix(0, nrow = K, ncol = m - 1)
HOIF0_folds <- matrix(0, nrow = K, ncol = m - 1)
IIFF1_folds <- matrix(0, nrow = K, ncol = m - 1)
IIFF0_folds <- matrix(0, nrow = K, ncol = m - 1)

for (j in 1:K) {
  # Define fold indices
  I_j <- which(fold_indices == j)
  I_not_j <- which(fold_indices != j)

  # Step 3: Compute Omega on training set (not I_j)
  Z_train <- Z[I_not_j, , drop = FALSE]
  A_train <- A[I_not_j]
  Omega_j <- compute_gram_inverse(Z_train, A_train, method = inverse_method)

  # Step 4: Compute basis matrices on test set (I_j)
  Z_test <- Z[I_j, , drop = FALSE]
  B_matrices_j <- compute_basis_matrix(Z_test, Omega_j$Omega1, Omega_j$Omega0)

  # Step 5: Compute HOIF on test set
  residuals_j <- list(
    R1 = residuals$R1[I_j],
    r1 = residuals$r1[I_j],
    R0 = residuals$R0[I_j],
    r0 = residuals$r0[I_j]
  )

  results_j <- compute_hoif_estimators(residuals_j, B_matrices_j, m, backend)

  # Store results
  ATE_folds[j, ] <- results_j$ATE
  HOIF1_folds[j, ] <- results_j$HOIF1
  HOIF0_folds[j, ] <- results_j$HOIF0
  IIFF1_folds[j, ] <- results_j$IIFF1
  IIFF0_folds[j, ] <- results_j$IIFF0
}

# Average across folds
results <- list(
  ATE = colMeans(ATE_folds),
  HOIF1 = colMeans(HOIF1_folds),
  HOIF0 = colMeans(HOIF0_folds),
  IIFF1 = colMeans(IIFF1_folds),
  IIFF0 = colMeans(IIFF0_folds),
  orders = 2:m
)
\end{verbatim}

\}

\# Add convergence plot data results\(convergence_data <- data.frame(
    order = results\)orders, ATE = results\$ATE )

class(results) \textless- ``hoif\_ate'' return(results) \}

\#' Print method for hoif\_ate objects \#' \#' @param x Object of class
hoif\_ate \#' @param \ldots{} Additional arguments (unused) \#' \#'
@export print.hoif\_ate \textless- function(x, \ldots) \{ cat(``HOIF
Estimators for Average Treatment Effect\n'')
cat(``=============================================\n\n'')

cat(``Estimates by order:\n'') print(data.frame( Order = x\(orders,
    ATE = round(x\)ATE, 4), HOIF1 = round(x\(HOIF1, 4),
    HOIF0 = round(x\)HOIF0, 4) ))

cat(``\nFinal ATE estimate (highest order):'', round(tail(x\$ATE, 1),
4), ``\n'') \}

\#' Plot convergence of ATE estimates \#' \#' @param x Object of class
hoif\_ate \#' @param \ldots{} Additional arguments passed to plot \#'
\#' @export plot.hoif\_ate \textless- function(x, \ldots) \{
plot(x\(orders, x\)ATE, type = ``b'', pch = 19, xlab = ``Order'', ylab =
``ATE Estimate'', main = ``Convergence of HOIF-ATE Estimator'', \ldots)
abline(h = tail(x\$ATE, 1), lty = 2, col = ``red'') grid() \}

\end{document}
